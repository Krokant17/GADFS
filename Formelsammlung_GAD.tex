\documentclass[12pt,a4paper]{article}
\usepackage[utf8]{inputenc}
\usepackage[german]{babel}
\usepackage[T1]{fontenc}
\usepackage{amsmath}
\usepackage{amsfonts}
\usepackage{amssymb}
\usepackage{xcolor}
\usepackage{colortbl}
\usepackage{graphicx}
\author{Jonas Schulz}
\begin{document}
\section{Komplexitätsanalyse}
\subsection{Effizienzmaße}
\begin{itemize}
\item Worst Case: $\color{blue}t(n)=max\{T(i):i\in I_n\}$
\item Average Case: \begin{itemize}
\item Gleichverteilt:\\
$\color{blue}t(n)=\frac{1}{\vert I_n \vert}\sum_{i\in I_n}T(i)$
\item Nicht gleichverteilt:\\
$\color{blue}t(n)=\sum_{i\in I_n}p_i\cdot T(i)$
\end{itemize}
\item Best Case: $\color{blue}t(n)=min\{T(i):i\in I_n\}$
\end{itemize}
\subsection{Formalisierung asymptotischen Verhaltens}
\begin{itemize}
\item $\color{red}\mathcal{O}\color{blue}(f(n)) = \{ g(n):\color{red}\exists\color{blue} c>0: \exists n_0>0: \forall n\geq n_0: g(n)\color{red}\leq\color{blue} c\cdot f(n)\}$
\item $\color{red}\Omega\color{blue}(f(n)) = \{ g(n):\color{red}\exists\color{blue} c>0: \exists n_0>0: \forall n\geq n_0: g(n)\color{red}\geq\color{blue} c\cdot f(n)\}$
\item $\color{red}\Theta \color{blue}(f(n)) = \color{red}\mathcal{O}\color{blue}(f(n)) \color{red}\cap\Omega \color{blue}(f(n))$
\item $\color{red}o\color{blue}(f(n)) = \{ g(n):\color{red}\forall\color{blue} c>0: \exists n_0>0: \forall n\geq n_0: g(n)\color{red}\leq\color{blue} c\cdot f(n)\}$
\item $\color{red}\omega\color{blue}(f(n)) = \{ g(n):\color{red}\forall\color{blue} c>0: \exists n_0>0: \forall n\geq n_0: g(n)\color{red}\geq\color{blue} c\cdot f(n)\}$
\end{itemize}
\subsection{Rechenregeln für $\mathcal{O}$-Notation}
\begin{itemize}
\item Polynome $k$-ten Grades $\in\Theta(n^k)$
\item Für Funktionen $f(n)$ bzw. $g(n)$ mit $\exists n_0 \forall n\geq n_0 : f(n) > 0$: \begin{itemize}
\item $c\cdot f(n)\in\Theta(f(n))$
\item $\mathcal{O}(f(n))+\mathcal{O}(g(n))=\mathcal{O}(f(n)+g(n))$
\item $\mathcal{O}(f(n))\cdot\mathcal{O}(g(n))=\mathcal{O}(f(n)\cdot g(n))$
\item $\mathcal{O}(f(n)+g(n))=\mathcal{O}(f(n))$, falls $g(n)\in\mathcal{O}(f(n))$
\item $\Omega(f(n))+\Omega(g(n))=\Omega(f(n)+g(n))$
\item $\Omega(f(n))\cdot\Omega(g(n))=\Omega(f(n)\cdot g(n))$
\item $\Omega(f(n)+g(n))=\Omega(f(n))$, falls $g(n)\in\mathcal{O}(f(n))$
\end{itemize}
\item Falls $f,g$ differenzierbar, $X$ eines der fünf Landau-Symbole:\\
 $f'(n)\in X(g'(n))\Rightarrow f(n)\in X(g(n))$
\end{itemize}
\section{Datenstrukturen}
\subsection{Dynamische Felder}
\begin{itemize}
\item Parameter:\begin{itemize}
\item $\color{cyan}\beta =2$: Wachstumsfaktor
\item $\color{cyan}\alpha =4$: max. Speicheroverhead
\item $\color{cyan}w=1$: momentane Feldgröße
\item $\color{cyan}n=0$: momentane Elementanzahl
\item $\color{cyan}b=new ~X[w]$: statisches Feld
\end{itemize}
\item Methoden:\begin{itemize}
\item get (int) : $X$ $\in\mathcal{O}(1)$
\item set (int, $X$) : void $\in\mathcal{O}(1)$
\item size () : int $\in\mathcal{O}(1)$
\item pushBack ($X$ $x$) : void $\in$ (reallocate ? $\mathcal{O}(reallocate)$ : $\mathcal{O}(1)$)
\item popBack () : void (Laufzeiten wie pushBack)
\item reallocate (int) : void $\in\mathcal{O}(n)$, amortalisiert $\in\mathcal{O}(1)$
\end{itemize}
\end{itemize}
\subsection{Doppelt verkettete Listen}

\subsection{Stacks und Queues}

\subsection{Binäre Heaps}
\begin{itemize}
\item -siftDown(int) : void $\in\mathcal{O}(log~n)$
\item -siftUp(int) : void $\in\mathcal{O}(log~n)$
\item +deleteMin() : X $\in\mathcal{O}(siftDown)$
\item +min() : X $\in\mathcal{O}(1)$
\item +insert(X) : void $\in\mathcal{O}(siftUp)$
\item +build($x_1,...,x_n$) : void $\in\mathcal{O}(n)$
\item +merge($B_2$) : void $\in\Theta(n)$
\end{itemize}
\subsection{Binomiale Heaps}
Binomialbäume: gleiche Operationen und Laufzeiten wie Binäre Heaps.\\
Weitere Operationen und Ausnahmen:
\begin{itemize}
\item decreaseKey $\in\mathcal{O}(log~n)$
\item remove $\in\mathcal{O}(log~n)$
\item merge $\in\mathcal{O}(log~n)$
\end{itemize}
\subsection{Fibonacci Heaps}
\begin{itemize}
\item min, insert, merge: $\mathcal{O}(1)$
\item decreaseKey: $\mathcal{O}(1)$ (amortalisiert)
\item deleteMin, remove: $\mathcal{O}(log~n)$ (amortalisiert)
\end{itemize}
\section{Hashing}
$c$-universelles Hashing:\\
Eine Familie $H$ von Hashfunktionen auf $\{0,...,m-1\}$ heißt $c$-universell, falls für jedes Paar $x\neq y$ von Schlüsseln gilt:\\
$\vert\{ h\in H: h(x)=h(y)\}\vert \leq \frac{c}{m}\vert H\vert$.\\
\begin{itemize}
\item $c$: Kollisionsmaß
\item $m$: Größe der Hashtabelle
\item $n$: Anzahl Elemente in der HT
\item Erwartete Laufzeit remove/find dann in $\mathcal{O}(1+c\cdot\frac{n}{m})$.
\end{itemize}
Mit $m$ prim und $h_a(x)=\langle a,x\rangle ~mod~m$ ist $H=\{h_a:a\in \{0,...,m-1\}^k\}$ 1-universell.\\
Mit $m$ prim und $h_a(x)=\sum_{i=1}^ka^{i-1}x_i ~mod~ m$ ist $H=\{h_a:a\in \{1,...,m-1\}\}$ $k$-universell.
\section{Sortieren}
\subsection{Master-Theorem}
Seien $a,b,c,d$ positive Konstanten und $n=b^k$ mit $k\in\mathbb{N}$.\\
Sei $r(n)=$\begin{itemize}
\item $\color{blue}a$, falls $\color{orange}n=1$,
\item $\color{blue}cn+d\cdot r(\frac{n}{b})$, falls $\color{orange}n>1$.
\end{itemize}
Dann gilt: $r(n)\in$\begin{itemize}
\item $\color{cyan}\Theta(n)$, falls $\color{yellow}d<b$,
\item $\color{cyan}\Theta(n~log~n)$, falls $\color{yellow}d=b$,
\item $\color{cyan}\Theta(n^{log_bd})$, falls $\color{yellow}d>b$.
\end{itemize}
\subsection{Verfahren}
\scalebox{.6}{
\begin{tabular}{|l|c|c|c|l|}
\hline
\multicolumn{1}{|c|}{Algorithmus} & \multicolumn{3}{|c|}{Laufzeit} & \multicolumn{1}{|c|}{Eigenschaften}\\ \cline{2-4}
 & Best Case & Average Case & Worst Case & \\
\hline
\hline
\color{purple}Selection Sort & \cellcolor{red!25}$\Theta(n^2)$ & \cellcolor{red!25}$\Theta(n^2)$ & \cellcolor{red!25}$\Theta(n^2)$ & Einfach zu implementieren\\
\hline
\color{purple}Bubble Sort & \cellcolor{red!25}$\mathcal{O}(n^2)$ & \cellcolor{red!25}$\mathcal{O}(n^2)$ & \cellcolor{red!25}$\mathcal{O}(n^2)$ & Einfach zu implementieren\\
\hline
\color{blue}Heap Sort & \cellcolor{green!25}$\mathcal{O}(n)$ & \cellcolor{green!25}$\mathcal{O}(n~log~n)$ & \cellcolor{green!25}$\mathcal{O}(n~log~n)$ & Var. d. Selection Sort, bessere Minimumstrategie\\
\hline
\color{purple}Insertion Sort & \cellcolor{green!25}$\mathcal{O}(n)$ & \cellcolor{red!25}$\mathcal{O}(n^2)$ & \cellcolor{red!25}$\mathcal{O}(n^2)$ & Einfach zu implementieren\\
\hline
\color{blue}Bogo Sort & \cellcolor{green!25}$\mathcal{O}(n)$ & \cellcolor{red!50}$\mathcal{O}(n\cdot n!)$ & \cellcolor{red!75}$\mathcal{O}(\infty)$ & Einfach zu implementieren\\
\hline
\color{purple}Shell Sort & \cellcolor{green!25}$\mathcal{O}(n)$ & $\mathcal{O}(n~log^2~n)$ & $\mathcal{O}(n~log^2~n)$ & Variante d. Insertion Sort, bessere Einfügestrategie\\
\hline
\color{red}Merge Sort & $\mathcal{O}(n~log~n)$ & \cellcolor{green!25}$\mathcal{O}(n~log~n)$ & \cellcolor{green!25}$\mathcal{O}(n~log~n)$ & Sehr gut für Multi-Threading geeignet\\
\hline
\color{blue}Quick Sort & $\mathcal{O}(n~log~n)$ & \cellcolor{green!25}$\mathcal{O}(n~log~n)$ & \cellcolor{red!25}$\mathcal{O}(n^2)$ & Schnellstes Verfahren in der Praxis \\
\hline
\color{red}Radix Sort &\cellcolor{cyan!25} $\mathcal{O}(d\cdot (n+k))$ & \cellcolor{cyan!25}$\mathcal{O}(d\cdot (n+k))$ & \cellcolor{cyan!25}$\mathcal{O}(d\cdot (n+k))$ & $d$  Stellen, $k$ mögl. 'Ziffern' pro Stelle\\
\hline
\end{tabular}}\\\\
\color{red} Stabil\\
\color{blue} In-Place\\
\color{purple} Stabil und In-Place
\color{black}
\section{Priority Queues}
\section{Suchstrukturen}
\section{Graphen}
\section{Pattern Matching}
\section{Datenkompression}
\end{document}